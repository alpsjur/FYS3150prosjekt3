\begin{abstract}
In this report,  Metropolis' algorithm is used for solving the two dimensional Ising model. By storing the energy and magnetization of the accepted states, thermodynamic quantities like the mean energy $\left\langle E \right\rangle $, mean absolute magnetisation $\left\langle |M| \right\rangle $, specific heat $C_V$ and susceptibility $\chi$ is calculated for different temperatures $T$ and lattice sizes $L\times L$. By looking at the lattice size $L=20$ for temperatures $T=1$ and $T=2.4$ with an unordered and ordered initial configuration of the lattice, I find that an ordered initial configuration gives a much shorter thermalisation time for the lower temperature. For the higher temperature, there seems to be no big difference in thermalisation time for the two starting configuration, with an thermalisation time of about $10^4$ Monte Carlo sweeps (mcs). I also find that for higher temperatures, new states are accepted more often, and there are more possible energy states. Examining the peak in $C_V$ as a function of $T$, and comparing different lattice sizes, I get an estimate of the critical temperature $T_C$ of $2.259\pm 0.015$ for a low temperature resolution run, and $2.363\pm 0.007$ for a high temperature resolution run, both within one standard deviation of the analytical value.  
\end{abstract}
