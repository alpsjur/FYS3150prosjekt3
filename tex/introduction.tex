\section{Introduction}
\label{sec:introduction}
One of the pillars of modern physics is statistical mechanics. It is crucial when studying a system with a large number of degrees of freedom, and can be used to understand the thermodynamic behaviour of a large system. By studying fluctuations around a value, one can connect microscopic states to macroscopic observations. Since an analytical solution involves all possible states of a system, in general, only small systems allow an analytical solution. However, numerical methods makes it possible to solve larger and more complex systems. Such a method is the Monte Carlo method, where random samples are used to determine the results. One of the most popular algorithms in the Monte Carlo family is the Metropolis' algorithm. 

Starting with the methods section, I will shortly introduce how some thermodynamic properties are calculated, before describing the Ising model and phase transitions in this model. I will then introduce the Metropolis' algorithm for solving the Ising model. Before the result section, a method for finding the probability distribution $p(E)$ and the implementation of the algorithm is described. In the results section, properties of the model are precented, before a time evolution of different thermodynamic properties are shown for a variety of lattice sizes, together with an estimate of the critical temperature in the thermodynamic limit of a infinite sized lattice. Finally, the results are discussed.    