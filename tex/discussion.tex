\section{Discussion}
\label{sec:discussion}
When a model is used in order to understand behaviour of a system, or for predicting certain quantities, it is important to understand how the model responds to different initial conditions, and how computations and parameters are connected. 
Considering Figure \ref{fig:EMvsMCS}, one can see a clear difference in the thermalisation time for $T=1.0$, depending on whether an ordered or unordered initial configuration is used. An ordered configuration, with all spins pointing in the same direction, has a lower energy than an unordered configuration. The expectation value for a system with a low temperature, and then also low energy, will therefore be closer to a system with an ordered configuration. 

As for the higher temperature $T=2.4$, the difference in thermalisation time for an ordered and unordered initial configuration is not that apparent when looking at Figure \ref{fig:EMvsMCS}. It is however clear that the expectation values fluctuates more for higher temperatures, than for the lower temperature with an ordered initial configuration. This can be connected with the probability distribution in Figure \ref{fig:prob} and the variance in Table \ref{tab:prob}. Also, the thermalisation time for the absolute magnetisation appears to be longer than for the energy, both for the higher temperature, and for the lower temperature with an unordered initial configuration. 

Based on these observations, an ordered initial state appears to be best suited for lower temperatures, while for higher temperatures both unordered and ordered initial configurations appears to give an equally good convergence rate.  

Looking at Figure \ref{fig:nAccepted}, one can see that there are in general more accepted new configuration for higher temperatures. Additionally, the number of accepted new states increases as $\sim$ mcs. An exception to this behaviour is the graph for $T=1.0$ with an unordered initial configuration, as the number of accepted configurations for few mcs is about the same as for the higher temperatures. This can be seen in relation to the long thermalisation time for this case. 

As mentioned, there is a connection between the fluctuations in Figure \ref{fig:EMvsMCS}, the probability distribution in Figure \ref{fig:prob} and the variance in Table \ref{tab:prob}. For higher temperatures, there is more and larger fluctuations in expectation values, and therefore more possible energy states, as seen in Figure \ref{fig:prob}. More possible energy states and a bigger spread in energy leads to a larger variance in the expectation values for higher temperatures. Comparing Figure \ref{fig:prob} and Table \ref{tab:prob}, one can also see that the expectation value for the energy is close to the maxima for the probability distribution. 


Finally, let's look at how well the model preforms in predicting the critical temperature in the thermodynamic limit $L\rightarrow \infty$. From Table \ref{tab:Tc}, one can see that both the low resolution and high resolution run gives a critical temperature within one standard deviation from the analytical value, and that the higher resolution gives a better approximation than the lower resolution. Further, taking the running mean does not improve the result of the higher resolution run.  