\section{Methods}
\label{sec:methods}

\subsection{Statistical physics and thermodynamics}


\subsection{The Ising model}
The Ising model consists of variables $s$ that can exist in two states, typically +1 or -1. These variables represent magnetic dipole moments of atomic spin, and are ordered in a two dimensional lattice consisting of $L\times L$ spins. Given that there is no magnetic field, the energy of the system is modelled as
\begin{equation}
E = -J\sum_{\left\langle ij\right\rangle } s_is_j,
\end{equation} 
where $J$ is a coupling constant expressing the strength of the interaction between neighbouring spins. The symbol $\left\langle ij\right\rangle$ means that the sum is over neighbouring spins. The magnetisation of the system is simply the sum of all the spins 
\begin{equation}
M = \sum_i s_i.
\end{equation}

\subsection{Monte Carlo simulations}

\subsection{Metropolis' algorithm}
One possibly way of solving the Ising model is by using the so called Metropolis' algorithm. 

